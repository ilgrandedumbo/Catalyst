\documentclass[12pt,a4paper]{article}

% ==== PACCHETTI DI BASE ====
\usepackage[english,italian]{babel}
\usepackage[lmargin=2.5cm,rmargin=2.5cm,tmargin=2.5cm,bmargin=2.5cm]{geometry}
\usepackage{hyperref}
\usepackage{xcolor}
\usepackage{graphicx}
\usepackage{caption}
\usepackage{subcaption}
\usepackage{minted} % Per blocchi di codice (se non serve, puoi toglierlo)
\usepackage[T1]{fontenc}
\usepackage{setspace}
\usepackage{csquotes}
\usepackage{longtable,booktabs,array}
\usepackage[
  backend=biber,
  style=apa
]{biblatex}

\addbibresource{assets/bib-template.bib}

\setstretch{1.3}

% ==== COLORI E LINK ====
\definecolor{LightGray}{gray}{0.9}
\hypersetup{
  colorlinks=true,
  linkcolor=blue,
  filecolor=maroon,
  citecolor=blue,
  urlcolor=blue
}

% ==== TITOLI ====
\title{Report Interviste}
\author{Nome del gruppo / Autore}
\date{\today}

\begin{document}
\maketitle
\tableofcontents
\newpage

% ==== METODOLOGIA ====
\section{Metodologia}
Le interviste sono state svolte durante la giornata del $\textbf{05/10/2025}$ al Market del Baratto, evento che unisce donazione e riuso di oggetti. (Agenti sul campo $\textbf{Chiara Santovito}$ e $\textbf{Niccolò Papini}$) 

Le domande si focalizzavano su:
\begin{itemize}
  \item [-] Motivazione che li ha spinti a riciclare la prima volta.
  \item [-] Modalità di categorizzazione del rifiuto (riciclabile o da buttare).
  \item [-] Motivi per il quale sono entrate in contatto con il Market.
  \item [-] Motivi che li spingono a scegliere il market e non i mezzi urbani (AMSA).
  \item [-] Motivi che li invogliano a tornare al Market.
\end{itemize}

L'intervista è stata svolta durante la giornata del $\textbf{08/10/2025}$ al nell'ufficio del responsabile dell'impresa ambientale. (Agente sul campo $\textbf{Giosia Pacilli}$) 

Le domande si focalizzavano su:
\begin{itemize}
  \item [-] Come funziona la gestione dei rifiuti urbani.
  \item [-] Limiti legali e burocraticità da rispettare.
\end{itemize}

\newpage


% ==== INTERVISTE ====
\section{Interviste}
Ogni sottosezione riporta una sintesi strutturata di un'intervista individuale.

\subsection{Intervista 1 – Beatrice 25 / Studentessa - Utente Diretto}
\textbf{Data:} 05 settembre 2025  
\textbf{Luogo / modalità:} Market del Baratto, dal vivo  

\paragraph{Esperienze e pratiche di riciclo}


Il primo contatto con pratiche di riciclo e scambio risale a circa sei-sette anni fa. Attualmente, entrambe frequentano il Tempio del Futuro Perduto per partecipare al Market del Baratto (Barattolo), un’iniziativa conosciuta attraverso i social network.
Le intervistate utilizzano inoltre Vinted come piattaforma digitale per il commercio e lo scambio di beni di seconda mano.

\paragraph{Tipologia di oggetti scambiati e motivazioni}

Gli oggetti maggiormente soggetti a scambio o riciclo sono principalmente vestiti e oggetti ingombranti ancora funzionanti. Le motivazioni alla base di tali pratiche includono:

\begin{itemize}
  \item la volontà di ridurre il consumismo.
  \item l’interesse per il riciclo e il riuso dei materiali.
  \item la necessità di liberare spazio domestico da oggetti inutilizzati.
\end{itemize}

\paragraph{Rapporto con i canali tradizionali di riciclo}

Entrambe le intervistate dichiarano di non utilizzare i servizi tradizionali di riciclo, quali AMSA o le isole ecologiche, poiché non ne hanno mai percepito il bisogno. Il baratto rappresenta per loro una soluzione sufficiente e più coerente con le proprie abitudini e valori.

\paragraph{Dimensione sociale e percezione del digitale}

Un elemento centrale nelle esperienze di Beatrice è la dimensione sociale dello scambio. Le intervistate affermano di preferire relazioni dirette e contatti umani all’interno di spazi pubblici.
Sebbene non escludano l’uso di strumenti digitali, dichiarano che non utilizzerebbero una piattaforma completamente online, a meno che essa consenta di individuare e connettersi con punti di scambio fisici, come i market del baratto.

\newpage

\subsection{Intervista 2 – Sabrina 25 e Valeria 35 / Studentesse e Lavoratrici - Utente Estremo e Utente Diretto}

\textbf{Data:} 05 settembre 2025  


\textbf{Luogo / modalità:} Market del Baratto / dal vivo 

\paragraph{Profilo delle intervistate}

Le intervistate sono \textbf{Sabrina} (25 anni), studentessa in Comunicazione e Marketing e lavoratrice nel settore web marketing, e \textbf{Valeria} (35 anni), funzionario pubblico.  
Sabrina possiede un livello avanzato di competenza tecnologica e informatica, mentre Valeria ha conosciuto il mondo dello scambio di beni grazie a Sabrina, sua cugina.

\paragraph{Esperienze e pratiche di riciclo}

Il primo contatto con il riciclo e lo scambio di beni è avvenuto in passato, anche se non è stata specificata una tempistica precisa. Entrambe utilizzano \textbf{Vinted} come piattaforma digitale per lo scambio o la vendita di oggetti di seconda mano.

\paragraph{Tipologia di oggetti scambiati e motivazioni}

Gli oggetti principali soggetti a scambio o riciclo comprendono:
\begin{itemize}
  \item vestiti;
  \item libri già letti;
  \item elettrodomestici ancora funzionanti.
\end{itemize}

Le motivazioni dichiarate dalle intervistate includono:
\begin{itemize}
  \item dare una seconda vita agli oggetti ben conservati che non vengono più utilizzati (ad esempio per cambio di città o mutamento dei gusti personali);
  \item liberare spazio domestico;
  \item valorizzare lo spirito di riuso e sostenibilità degli oggetti.
\end{itemize}

\paragraph{Rapporto con i canali tradizionali di riciclo}

Le intervistate dichiarano di non utilizzare servizi tradizionali di riciclo, come AMSA o le isole ecologiche, in quanto non ne hanno percepito la necessità. L’approccio delle intervistate si avvicina più alla donazione che a uno scambio formale.

\paragraph{Esperienza digitale e percezione della comodità}

Sebbene apprezzino le piattaforme digitali per lo scambio, le intervistate evidenziano alcune criticità relative alla comodità e al prezzo di certi beni, soprattutto nel caso di negozi che trattano articoli vintage o di seconda mano, considerati talvolta eccessivamente costosi.

\paragraph{Aspetti chiave dello scambio}

Le intervistate sottolineano alcuni elementi ritenuti fondamentali nello scambio di beni:
\begin{itemize}
  \item garanzia delle condizioni dei beni;
  \item formalità e correttezza da entrambe le parti;
  \item riduzione del rischio di frodi, soprattutto per articoli di marca o materiali specifici;
  \item accettazione di sistemi basati su token per incentivare scambi e donazioni, ad esempio offrendo sconti su musei o servizi culturali.
\end{itemize}

\newpage

\subsection{Intervista 3 – Luca 24 / Membro Market del Baratto - Utente Guida}
\textbf{Data:} 05 settembre 2025  
\textbf{Luogo / modalità:} Market del Baratto, dal vivo  

\paragraph{Profilo dell’intervistato}

L’intervistato, \textbf{Luca} (24 anni), ex studente di Design presso il Politecnico, è membro attivo del Market del Baratto (Barattolo). Ha iniziato come barista all’interno del market e oggi ricopre il ruolo di designer. Luca è inoltre familiare con l’informatica.  

\paragraph{Esperienze e pratiche di riciclo}

Luca ha maturato un’esperienza personale significativa nel baratto e nel riciclo. Gli oggetti principali soggetti a scambio o riciclo comprendono:
\begin{itemize}
  \item vestiti;
  \item libri già letti;
  \item elettrodomestici ancora funzionanti.
\end{itemize}
Per vestiti e elettrodomestici, prima di gettarli Luca verifica sempre se esiste una \textbf{modalità alternativa di utilizzo o riciclo}.  

\paragraph{Motivazioni}

Luca sottolinea l’importanza del concetto di baratto come manifesto e come principio guida:  

\begin{quote}
- "E comunque, come manifesto, come claim, diciamo è giustissimo nel senso cercare di riutilizzare quello che possiamo, visto che siamo sommersi da cose che non sono solo vestiti, ma in generale cercare di ri-abituare un po' le persone al baratto, quando non è necessario, per forza acquistare cose nuove."
\end{quote}

\paragraph{Comodità desiderata e criticità}

Luca evidenzia alcune esigenze specifiche per migliorare il funzionamento del market:
\begin{itemize}
  \item Implementare una \textbf{chat di ascolto richieste}, in quanto le FAQ risultano insufficienti;  
  \item Chiarezza sulla finalità e sul messaggio delle attività;  
  \item Strumenti per una gestione più efficace dei magazzini, con comunicazione chiara alle persone (es. tabelle giornaliere con informazioni).  
\end{itemize}

Una criticità segnalata riguarda la sicurezza: alcune persone si presentano con l’intento di rubare. Luca propone la creazione di una **tessera di accesso** per il controllo degli ingressi e maggiore sicurezza.  

\paragraph{Rapporto con enti esterni}

Luca dichiara di non aver mai collaborato con AMSA o altre grandi aziende, in quanto il progetto è completamente gratuito. Tuttavia, alcune persone usufruiscono dei servizi del market per sbarazzarsi di oggetti che non riescono a gestire nei propri magazzini.  

\newpage

\subsection{Intervista 4 – Pietro 54 anni / Utente Guida}
\textbf{Data:} 08 settembre 2025  
\textbf{Luogo / modalità:} Ufficio di Pietro / dal vivo 

\paragraph{Profilo dell’intervistato}

L’intervistato, \textbf{Pietro}, è responsabile di un’azienda del settore igiene ambientale da 25 anni.  

\paragraph{Esperienze e pratiche di raccolta rifiuti}

Pietro descrive il funzionamento del servizio di raccolta dei rifiuti ingombranti porta a porta, disponibile solo se il comune aderisce al servizio. In tal caso, l’operatore recupera esclusivamente i rifiuti indicati in una lista ufficiale; eventuali altri oggetti esposti al di fuori di questa lista costituiscono conferimento illegale. I rifiuti raccolti vengono trasportati in aree ecologiche e smaltiti separatamente nei cassoni.  

Nonostante una parte dei rifiuti ingombranti possa essere riutilizzata, le normative impediscono agli operatori di prelevare oggetti esposti al di fuori della lista ufficiale. Inoltre, per motivi logistici, alcuni rifiuti vengono tagliati o danneggiati durante il trasporto, con conseguente **opportunità persa di riciclo**.  

\paragraph{Difficoltà riscontrate}

Pietro evidenzia diverse difficoltà legate alla raccolta dei rifiuti ingombranti:  
\begin{itemize}
  \item Alcuni comuni non offrono il servizio, generando difficoltà logistica per i cittadini;  
  \item Preparazione dei rifiuti secondo le regole di sicurezza (dimensioni, peso) richiesta per la raccolta;  
  \item Distruzione o danneggiamento dei rifiuti ingombranti durante il trasporto, riducendo le possibilità di riutilizzo.  
\end{itemize}

\paragraph{Soluzioni e opportunità}

Secondo l’esperto, gli oggetti ingombranti ancora utilizzabili non dovrebbero essere considerati rifiuti, ma oggetti di seconda mano che possono essere venduti o scambiati sul mercato. In alternativa, i materiali che compongono gli oggetti (plastica, vetro, componenti ancora funzionanti come motori) possono essere riciclati.  

\paragraph{Cambiamenti necessari}

Pietro sottolinea la necessità di un cambiamento nella $\textbf{mentalità}$ $\textbf{dell’utente}$, poiché spetta al cittadino decidere se un oggetto sia un rifiuto o un bene di seconda mano. Incentivare la cultura del riuso può favorire il recupero degli oggetti ancora funzionali.  

\paragraph{Esperienze di riferimento}

L’intervistato cita l’esempio di Copenhagen, dove il riciclo e il riuso sono diffusi tra i cittadini: con pochi soldi è possibile arredare un appartamento con oggetti di seconda mano in buono stato. Secondo Pietro, tale pratica fa parte della cultura e della mentalità della popolazione, e rappresenta un modello da cui trarre spunto per incentivare il cambiamento nella gestione dei rifiuti.  

\newpage

% ==== CONCLUSIONI ====

\section{Analisi complessiva delle interviste e dei dati del form}

\subsection{Profilo dei partecipanti}

Dall’analisi delle interviste emerge che la maggior parte dei partecipanti è costituita da \textbf{giovani studenti} (\textbf{81,4\%}). La popolazione intervistata comprende anche lavoratori adulti e utenti esperti del settore igiene ambientale o organizzatori di mercati di scambio.  

Il \textbf{70\% dei partecipanti} dichiara di aver già sperimentato attività di \textbf{riciclo, riuso o riparazione} nella propria vita. Le opzioni più utilizzate comprendono principalmente la \textbf{riparazione} degli oggetti e la \textbf{vendita/scambio} tramite piattaforme digitali o mercati fisici.  

\subsection{Motivazioni e comportamenti}

Le principali motivazioni emerse dalle interviste e dal form includono:  
\begin{itemize}
    \item \textbf{Comodità} nell’accesso a soluzioni di riuso o scambio;  
    \item \textbf{Risparmio economico}, soprattutto rispetto ai costi di riparazione o acquisto di un oggetto nuovo;  
    \item \textbf{Sostenibilità e riduzione del consumismo}, in particolare tra gli studenti e gli utenti dei market del baratto.  
\end{itemize}

Quando gli oggetti vengono scartati, i motivi principali sono \textbf{costi elevati di riparazione} o \textbf{qualità dubbia dopo la riparazione}. Il supporto alla riparazione sembra incentivare maggiormente il riuso e la conservazione degli oggetti.  

\subsection{Modalità di scambio e vendita}

Per quanto riguarda gli scambi o le vendite dal vivo, emerge una preferenza per:  
\begin{itemize}
    \item la verifica dell’identità degli utenti coinvolti;  
    \item l’utilizzo di \textbf{luoghi fisici affidabili} come negozi o mercati organizzati.  
\end{itemize}

Le piattaforme digitali più utilizzate sono \textbf{Vinted} e altri e-market simili, in cui la vendita o lo scambio richiede comunque attenzione al profilo dell’acquirente o scambi con modalità controllate.  

\subsection{Accesso ai servizi tradizionali di riciclo}

Nonostante molti utenti dispongano di \textbf{auto}, l’accesso alle isole ecologiche è spesso limitato: alcuni non ne conoscono l’ubicazione, mentre per altri la distanza varia tra \textbf{2 e 5 km}. Questo rappresenta un ostacolo al riciclo tradizionale e favorisce il ricorso a piattaforme digitali o mercati fisici.  

\subsection{Esperienze personali recenti}

Alcune esperienze segnalate dai partecipanti evidenziano sia successi sia ostacoli nella pratica del riuso e della riparazione:  

\begin{itemize}
    \item ``Ho riparato il mio aspirapolvere a casa per economicità; ora funziona. Non l'ho portato nei centri appositi perché il costo sarebbe stato troppo elevato rispetto all'utilità.''  
    \item ``Sono pigro.''  
    \item ``Il riuso mi ha permesso di risparmiare sui costi dell'oggetto, ma l’efficienza non era la stessa.''  
    \item ``Venduto su Vinted/Subito. Prima ho controllato il profilo del compratore, poi ho spedito o venduto a mano. C'è un po' di apprensione prima di vendere/scambiare, ma è bello dare una seconda vita agli oggetti e magari ottenere qualche soldo.''  
    \item ``Truffe.''  
\end{itemize}

Queste esperienze confermano le osservazioni principali sulle motivazioni, le difficoltà e le precauzioni adottate dagli utenti nel riuso e nello scambio degli oggetti.  

\subsection{Sintesi}

In sintesi, dai dati raccolti emerge un quadro in cui:  
\begin{itemize}
    \item i giovani studenti costituiscono la fascia più attiva nel riuso e nello scambio;  
    \item le piattaforme digitali sono ampiamente utilizzate, ma il contatto fisico resta importante per la fiducia;  
    \item la comodità, il risparmio economico e la sostenibilità sono motivazioni principali;  
    \item le difficoltà maggiori riguardano la logistica, la sicurezza e la valutazione della qualità degli oggetti.  
\end{itemize}
Questi elementi possono guidare eventuali progetti futuri per migliorare i servizi di riciclo, il supporto alla riparazione e l’organizzazione di mercati fisici o digitali di scambio.  

\end{document}
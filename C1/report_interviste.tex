\documentclass[12pt,a4paper]{article}

% ==== PACCHETTI DI BASE ====
\usepackage[english,italian]{babel}
\usepackage[lmargin=2.5cm,rmargin=2.5cm,tmargin=2.5cm,bmargin=2.5cm]{geometry}
\usepackage{hyperref}
\usepackage{xcolor}
\usepackage{graphicx}
\usepackage{caption}
\usepackage{subcaption}
\usepackage{minted} % Per blocchi di codice (se non serve, puoi toglierlo)
\usepackage[T1]{fontenc}
\usepackage{setspace}
\usepackage{csquotes}
\usepackage{longtable,booktabs,array}
\usepackage[
  backend=biber,
  style=apa
]{biblatex}

\addbibresource{assets/bib-template.bib}

\setstretch{1.3}

% ==== COLORI E LINK ====
\definecolor{LightGray}{gray}{0.9}
\hypersetup{
  colorlinks=true,
  linkcolor=blue,
  filecolor=maroon,
  citecolor=blue,
  urlcolor=blue
}

% ==== TITOLI ====
\title{Report Interviste}
\author{Nome del gruppo / Autore}
\date{\today}

\begin{document}
\maketitle
\tableofcontents
\newpage

% ==== INTRODUZIONE ====
\section{Introduzione}
Questo documento raccoglie e sintetizza le interviste condotte nell'ambito del progetto \textit{[inserire nome progetto]}.  
L’obiettivo è analizzare i punti di vista degli intervistati, evidenziare temi ricorrenti e trarre conclusioni utili per la fase successiva del lavoro.

% ==== METODOLOGIA ====
\section{Metodologia}
Le interviste sono state realizzate tra il \textbf{[data inizio]} e il \textbf{[data fine]}, seguendo una metodologia qualitativa.  
Sono stati selezionati \textbf{N} partecipanti appartenenti a diverse categorie (specificare ruoli o profili).  
Le domande si focalizzavano su:
\begin{itemize}
  \item [Esempio 1] Comprensione del contesto di utilizzo;
  \item [Esempio 2] Esperienza d’uso di strumenti attuali;
  \item [Esempio 3] Necessità e aspettative future.
\end{itemize}

% ==== INTERVISTE ====
\section{Sintesi delle Interviste}
Ogni sottosezione riporta una sintesi strutturata di un'intervista individuale.

\subsection{Intervista 1 – Nome intervistato / ruolo}
\textbf{Data:} 12 settembre 2025  
\textbf{Luogo / modalità:} Online su Zoom  
\textbf{Durata:} 45 minuti

\paragraph{Domande chiave}
\begin{enumerate}
  \item Qual è il suo ruolo principale nell'organizzazione?
  \item Quali difficoltà incontra più spesso nel suo lavoro?
  \item Come valuta gli strumenti attualmente in uso?
\end{enumerate}

\paragraph{Sintesi delle risposte}
L’intervistato ha evidenziato che [...].  
Punti principali emersi:
\begin{itemize}
  \item Necessità di migliorare [...]
  \item Apprezzamento per [...]
  \item Suggerimenti: [...]
\end{itemize}

\subsection{Intervista 2 – Nome intervistato / ruolo}
(Stesso formato della precedente.)

% ==== ANALISI TRASVERSALE ====
\section{Analisi Trasversale}
In questa sezione vengono messi a confronto i risultati delle interviste, individuando temi comuni e divergenze.  
\begin{itemize}
  \item Tema 1: ...
  \item Tema 2: ...
  \item Differenze di opinione: ...
\end{itemize}

% ==== CONCLUSIONI ====
\section{Conclusioni}
Riepilogo finale delle evidenze principali e delle implicazioni per il progetto.

% ==== BIBLIOGRAFIA ====
\newpage
\printbibliography

\end{document}